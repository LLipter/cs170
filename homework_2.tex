% Search for all the places that say "PUT SOMETHING HERE".

\documentclass[11pt]{article}
\usepackage{amsmath,textcomp,amssymb,geometry,graphicx,enumerate}

\def\Name{Ran Liao}  % Your name
\def\SID{3034504227}  % Your student ID number
\def\Homework{2} % Number of Homework
\def\Session{Spring 2019}


\title{CS170--Spring 2019 --- Homework \Homework\ Solutions}
\author{\Name, SID \SID}
\markboth{CS170--\Session\  Homework \Homework\ \Name}{CS170--\Session\ Homework \Homework\ \Name}
\pagestyle{myheadings}
\date{}

\newenvironment{qparts}{\begin{enumerate}[{(}a{)}]}{\end{enumerate}}
\def\endproofmark{$\Box$}
\newenvironment{proof}{\par{\bf Proof}:}{\endproofmark\smallskip}

\textheight=9in
\textwidth=6.5in
\topmargin=-.75in
\oddsidemargin=0.25in
\evensidemargin=0.25in


\begin{document}
\maketitle
Collaborators: YiLin Wu, YeFan Zhou

\section{Study Group}
\begin{tabular}{ll}
    Name        &   SID         \\\hline
    Ran Liao    &   3034504227  \\  
    YiLin Wu    &   3034503863  \\
    YeFan Zhou  &   3034503824 
\end{tabular}

\newpage
\section{Asymptotic Complexity Comparisons}
\begin{qparts}
\item 
$3 \leq  7 \leq 2 \leq 5 \leq 4 \leq 9 \leq 8 \leq 6 \leq 1$

\item 
\renewcommand{\theenumii}{\roman{enumii}}
\begin{enumerate}
  \item $\log_3n = \Theta(\log_4n)$
  
  Proof:
  \[
	 \log_3n  =  \log_{(4^{\log_4 3})}n = (\frac{1}{\log_4 3})\log_4n = \Theta(\log_4n)
 \]
  \item $n\log(n^4) = O(n^2\log(n^3))$
  
  Proof:
 \[
  	\lim_{n \to +\infty} \frac{n^2\log(n^3)}{n\log(n^4)} = \lim_{n \to +\infty} \frac{3n^2\log n}{4n\log n} = \lim_{n \to +\infty} \frac{3}{4}n = \infty
 \]
 
 \item $\sqrt{n} = \Omega((\log n)^3)$
 
 Proof: (Use L\textquotesingle Hôpital's rule)
 \begin{align*}
 	\lim_{n \to+\infty} \frac{\sqrt{n}}{(\log n)^3} 
	&= \lim_{n \to+\infty} \frac{\frac{1}{2\sqrt{n}}}{3(\log n)^2\frac{1}{n}} = \lim_{n \to+\infty} \frac{\sqrt{n}}{(\log n)^2} \\
	&= \lim_{n \to+\infty} \frac{\frac{1}{2\sqrt{n}}}{2(\log n)\frac{1}{n}} = \lim_{n \to+\infty} \frac{\sqrt{n}}{\log n} \\
	&= \lim_{n \to+\infty} \frac{\frac{1}{2\sqrt{n}}}{\frac{1}{n}} = \lim_{n \to+\infty} \sqrt{n} = \infty
 \end{align*}
 
 \item $2^n = \Theta(2^{n+1})$
 
 Proof:
 \[
 	2^{n+1} = 2 * 2^n = \Theta(2^n)
 \]
 
 \item $n = \Omega((\log n)^{\log \log n})$
 
 Proof:(Use L\textquotesingle Hôpital's rule)
 
 Take the logarithm of both functions
 \[
 	\log (\log n)^{\log \log n} = (\log \log n)^2
 \]
 Therefore, it is equivalent to compare $(\log \log n)^2$ and $\log n$
 \begin{align*}
 	\lim_{n \to+\infty} \frac{(\log \log n)^2}{\log n} 
	&= \lim_{n \to+\infty} \frac{2(\log \log n)\frac{1}{\log n}\frac{1}{n}}{\frac{1}{n}} = \lim_{n \to+\infty}\frac{2(\log \log n)}{\log n} \\
	&= \lim_{n \to+\infty} \frac{\frac{1}{\log n}\frac{1}{n}}{\frac{1}{n}} = \lim_{n \to+\infty}\frac{1}{\log n} = 0
 \end{align*}
 
 \item $n + \log n = \Theta(n + (\log n)^2)$
 
 Proof:(Use L\textquotesingle Hôpital's rule)
  \begin{align*}
  	\lim_{n \to+\infty} \frac{n + \log n}{n + (\log n)^2}
		&= \lim_{n \to+\infty} \frac{1 + \frac{1}{n}}{1 + 2(\log n)\frac{1}{n}} = \lim_{n \to+\infty}\frac{n +1}{n + 2\log n} \\
		&= \lim_{n \to+\infty} \frac{1}{1 + \frac{2}{n}} = 1
   \end{align*}
   
  \item $\log (n!) = O(n\log n)$
  
  Proof: Since $n\log n = log (n^n)$, it's equivalent to compare $n!$ and $n^n$. Obviously, $n! = O(n^n)$.
 
\end{enumerate}

\end{qparts}

\newpage
\section{In Between Functions}

\newpage
\section{Bit Counter}

\begin{tabular}{l | c}
    n	&   \# Total Flips  \\\hline
    1  &   1  \\  
    2  &   4  \\
    3  &   11 \\
    4  &   26 \\
\end{tabular}

There're two stages when counting from 0 to $2^n-1$ for a n-bit counter. 

\begin{itemize}
	\item Stage1 : when most-significant bit is 0
	\item Stage2 : when most-significant bit is 1
\end{itemize}

Suppose $f(n)$ represents the total number of flips need for n-bit long counter. Both in stage 1 and 2, $f(n-1)$ flips is needed to flip all bits to 1 except the most-significant bit. When switching between stage 1 and 2, another $n$ flips is introduced. 1 flip to switch the most-significant bit from 0 to 1. $n-1$ flips needed to switch all other bits to 0. Therefore, we have the following recursive formula:
\[
	f(n) = 2f(n-1) + n
\]
To solve this formula, we keep using it recursively, providing $f(1) = 1$ as base case:
 \begin{align}
 	f(n) 	&= 2f(n-1) + n \nonumber \\
		&= 4f(n-2) + 2(n-1) + n \nonumber \\
		&= 8f(n-3) + 4(n-2) + 2(n-1) + n \nonumber \\
		&= 2^kf(n-k) + 2^{k-1}(n-k+1) + \ldots + 2^2(n-2) + 2^1(n-1) + 2^0(n-0) \nonumber \\
		&= 2^{n-1}f(1) + 2^{n-2}(2) + \ldots + 2^2(n-2) + 2^1(n-1) + 2^0(n-0) \nonumber  \\
		&= 2^{n-1}(1) + 2^{n-2}(2) + \ldots + 2^2(n-2) + 2^1(n-1) + 2^0(n-0) \label{eq1} 
 \end{align}
Then we multiply it by 2:
 \begin{align}
	2f(n) =  2^{n}(1) + 2^{n-1}(2) + \ldots + 2^3(n-2) + 2^2(n-1) + 2^1(n-0) \label{eq2}  
 \end{align}
By \eqref{eq2} - \eqref{eq1}, we have
 \begin{align*}
 	2f(n) - f(n) &= 2^n + 2^{n-1} + \ldots + 2^1 -n \\
	f(n)		&= \frac{2(1-2^n)}{1-2} - n \\
			&= 2^{n+1} - n - 2 \\
			&= \Theta(2^n)
  \end{align*}
  
  \newpage
  \section{Recurrence Relations}
 Master theorem:
 
Suppose $a, b, c \in \mathbb{R}^+$, $b > 1$ and $T(1)=1$

Given
 \begin{equation*}
 T(n) = aT(\frac{n}{b}) + \Theta(n^c),
 \end{equation*}
 
 We have
\[
 T[n]=\left\{
\begin{array}{lcl}
\Theta(n^{\log_ba})      &      & c < \log_ba\\
\Theta(n^c\log_2n)    &      & c = \log_ba\\
\Theta(n^c)     &      & c > \log_ba\\
\end{array} \right. 
\]

\renewcommand{\labelenumi}{(\alph{enumi})}
\begin{enumerate}
	\item 
	Let $a=4, b=2, c=1$. Since $\log_ba = log_2 4 = 2 > c$, $T(n) = \Theta(n^{\log_ba}) = O(n^2)$.
	\item 
	Let $a=4, b=3, c=2$. Since $\log_ba = log_3 4 < c$, $T(n) = \Theta(n^c) = O(n^2)$.
	\item
	\begin{align*}
		T(n) &= T(\sqrt{n}) + 1 \\
			&= T(n^\frac{1}{2^1}) + 1 \\
			&= T(n^\frac{1}{2^2}) + 1 + 1 \\
			&= T(n^\frac{1}{2^k}) + k \\
	\end{align*}
	This recursion process ends when $n^\frac{1}{2^k}$ reaches 2.
	\begin{align*}
		n^\frac{1}{2^k} &= 2 \\
		\frac{1}{2^k} &= \log_n2  \\
		2^k &= \log_2n \\
		k &= \log_2 \log_2n  \\
	\end{align*}
	Therefore, $T(n) = \Theta(\log \log n)$
\end{enumerate}


  \newpage
  \section{Hadamard matrices}
  
\renewcommand{\labelenumi}{(\alph{enumi})}
\begin{enumerate}
	\item 
	\[
		H_0=
		\begin{bmatrix} 
			1 \\
		\end{bmatrix}
	\]
	\[
		H_1=
		\begin{bmatrix} 
			1 & 1 \\ 
			1 & -1 
		\end{bmatrix}
	\]
		\[
		H_2=
		\begin{bmatrix} 
			1 & 1 & 1 & 1 \\ 
			1 & -1 & 1 & -1 \\ 
			1 & 1 & -1 & -1 \\
			1 & -1 & -1 & 1 \\
		\end{bmatrix}
	\]
	
	\item 
	\[
		H_2v = 
		\begin{bmatrix} 
			1 & 1 & 1 & 1 \\ 
			1 & -1 & 1 & -1 \\ 
			1 & 1 & -1 & -1 \\
			1 & -1 & -1 & 1 \\
		\end{bmatrix}
		\begin{bmatrix} 
			1  \\ 
			-1  \\ 
			-1  \\
			1  \\
		\end{bmatrix}
		=
		\begin{bmatrix} 
			0  \\ 
			0  \\ 
			0  \\
			4  \\
		\end{bmatrix}
	\]
	
	\item 
	\begin{align*}
		u_1 &= H_1(v_1 + v_2) \\
			&= 
				\begin{bmatrix} 
					1 & 1 \\ 
					1 & -1 
				\end{bmatrix}
				(
				\begin{bmatrix} 
					1  \\ 
					-1 \\
				\end{bmatrix}
				+
				\begin{bmatrix} 
					-1  \\ 
					1 \\
				\end{bmatrix}	
				)	\\	
			&= 	
				\begin{bmatrix} 
					1 & 1 \\ 
					1 & -1 
				\end{bmatrix}
				\begin{bmatrix} 
					0  \\ 
					0 \\
				\end{bmatrix}	\\
			&= 
				\begin{bmatrix} 
					0  \\ 
					0 \\
				\end{bmatrix}					
	\end{align*}
	\begin{align*}
		u_2 &= H_1(v_1 - v_2) \\
			&= 
				\begin{bmatrix} 
					1 & 1 \\ 
					1 & -1 
				\end{bmatrix}
				(
				\begin{bmatrix} 
					1  \\ 
					-1 \\
				\end{bmatrix}
				-
				\begin{bmatrix} 
					-1  \\ 
					1 \\
				\end{bmatrix}	
				)	\\	
			&= 	
				\begin{bmatrix} 
					1 & 1 \\ 
					1 & -1 
				\end{bmatrix}
				\begin{bmatrix} 
					2  \\ 
					-2 \\
				\end{bmatrix}	\\
			&= 
				\begin{bmatrix} 
					0  \\ 
					4 \\
				\end{bmatrix}					
	\end{align*}
	\[
		u =
		\begin{bmatrix}
			u_1 \\
			u_2 \\
		\end{bmatrix}
		=
		\begin{bmatrix}
			0 \\
			0 \\
			0 \\
			4 \\
		\end{bmatrix}		
	\]
	
	$u$ is identical to $H_2v$
	
	\item
	\begin{align*}
		H_kv &= 			
				\begin{bmatrix} 
					H_{k-1} & H_{k-1} \\ 
					H_{k-1} & -H_{k-1} \\ 
				\end{bmatrix}
				\begin{bmatrix}
					v_1 \\
					v_2 \\
				\end{bmatrix} \\
			&= 
				\begin{bmatrix} 
					H_{k-1}v_1 + H_{k-1} v_2 & H_{k-1} v_1 + H_{k-1} v_2\\ 
					H_{k-1}v_1 + H_{k-1} v_2&  -H_{k-1}v_1 - H_{k-1} v_2 \\ 
				\end{bmatrix}	\\
			&= 		
				\begin{bmatrix} 
					H_{k-1}(v_1 + v_2) & H_{k-1}(v_1 + v_2) \\ 
					H_{k-1}(v_1 + v_2) &  -H_{k-1}(v_1 + v_2)  \\ 
				\end{bmatrix}	
	\end{align*}
\end{enumerate}

\end{document}