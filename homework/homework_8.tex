% Search for all the places that say "PUT SOMETHING HERE".

\documentclass[11pt]{article}
\usepackage{amsmath,textcomp,amssymb,geometry,graphicx,enumerate}

\def\Name{Ran Liao}  % Your name
\def\SID{3034504227}  % Your student ID number
\def\Homework{8} % Number of Homework
\def\Session{Spring 2019}


\title{CS170--Spring 2019 --- Homework \Homework\ Solutions}
\author{\Name, SID \SID}
\markboth{CS170--\Session\  Homework \Homework\ \Name}{CS170--\Session\ Homework \Homework\ \Name}
\pagestyle{myheadings}
\date{\today}

\newenvironment{qparts}{\begin{enumerate}[{(}a{)}]}{\end{enumerate}}
\def\endproofmark{$\Box$}
\newenvironment{proof}{\par{\bf Proof}:}{\endproofmark\smallskip}

\textheight=9in
\textwidth=6.5in
\topmargin=-.75in
\oddsidemargin=0.25in
\evensidemargin=0.25in


\begin{document}
\maketitle
Collaborators:Jingyi Xu, Renee Pu

\section{Study Group}
	\begin{tabular}{ll}
		Name		&   SID         		\\\hline
		Ran Liao		&   3034504227  	\\  
		Jingyi Xu		&   3032003885  	\\
		Renee Pu		&   3032083302  	\\
	\end{tabular}

	



\newpage
\section{A Dice Game}
\begin{qparts}
	
	\item 
	
	\[
		W(x, y, z) = \max \{\frac{1}{6} (1 - W(y, x+1, 0)) + \sum_{i=2}^6 \frac{1}{6} W(x, y, z + i), (1 - W(y, x+\max(z, 1), 0)) \}		
	\]
	
	\item 
	
	\renewcommand{\theenumii}{\roman{enumii}}
	\begin{enumerate}
		\item \textbf{Main Idea}
		
		Alice have two choices, i.g. roll dice or pass. So the probability represented by $W(x, y, z)$ should be maximum number resulted by these two choices. And if her choice results in ending her turn, it's then Bob's turn. So we can represents the winning probability by 1 minus Bob's winning probability, since we assume both of them play optimally.
		
		The recursive relation is presented in the previous part. Run dynamic programming algorithm to solve problem according to it.
		
		\item \textbf{Proof of Correctness}
		
		If Alice choose to roll another dice, she has chance of $\frac{1}{6}$ to get 1 and end her turn, which is represented by $\frac{1}{6} (1 - W(y, x+1, 0))$. Or she may roll other number, which is represented by $\sum_{i=2}^6 \frac{i}{6} W(x, y, z + i)$. If Alice choose to pass, she will get $z$ points and end her turn, which is represented by $1 - W(y, x+ z, 0)$.
		
		\item \textbf{Runtime Analysis}
		
		There're $N^3$ subproblems and each of them will only cost constant time. Therefore, the total runtime is $O(N^3)$.
	
	\end{enumerate}
		
\end{qparts}



\newpage
\section{Knightmare}
\begin{qparts}
	\item \textbf{Main Idea}
	
	Denote $L(i)$ as a set that $j$ is in it if there's a knight in position $(i, j)$.
	
	Define subproblem as $T[i][L(i)][L(i-1)]$, which represents the number of ways to put knights in the first $i$ rows given the knights setting in row $i$ and row $i-1$ as $L(i)$ and $L(i-1)$.
	
	The recursive relation is as follows.
	\[
		T[i][L(i)][L(i-1)] = \sum_{\{L(i-2) | \text{ knights in } L(i-2) \text{ cannot attack knights in }L(i) \}}(T[i-1][L(i-1)][L(i-2)])
	\]
	
	
	 And the final solution is $\sum_{L(i)}\sum_{L(i-1)}T[N][L(i)][L(i-1)]$.

	\item \textbf{Proof of Correctness}
	
	The idea is that only knights in row $i-2$ can attack knights in row $i$. So given a specific setting of knights in row $i$. I just need to iterate through all possible combination of settings in row $i-1$ and $i-2$. And sum those number together will give the optimal number for $T[i][L(i)][L(i-1)]$.
	
	\item \textbf{Runtime Analysis}
	
	Each row has at most $2^M$ situations to put knights. 
	So, there're $N(2^M)^2$ subproblems and each of them will cost $O(2^M)$ time to check each possible situation. Therefore, the overall runtime is $O(N(2^M)^3) = O(N2^{3M})$.

\end{qparts}




\newpage
\section{Triangulating a polygon}
\begin{qparts}
	\item \textbf{Main Idea}
	
	First sort all points into clockwise order and denote $D[i][j]$ as the distance between $i$th point and $j$th point, i.g. $D[i][j] = \sqrt{(x_i - x_j)^2 + (y_i - y_j)^2}$.
	
	Define subproblem as $T[i][j]$, which represents the optimal solution to triangulate polygon defined by $i$th point to $j$th point.
	The recursive relation is as follows. In base case, if $j - i < 3$, $T[i][j]$ is 0.
	\[
		T[i][j] = \min \{ \min_{k=i+2}^{j-2}\{ T[i][k] + T[k][j] + D[i][k] + D[j][k] \}, T[i][j-1] + D[i][j-1], T[i+1][j], + D[i+1][j]\}    
	\]
	
	
	 And the final solution is $T[1][n]$.

	\item \textbf{Proof of Correctness}
	
	In base case, triangle do not need to be triangulated, so the cost is 0. In induction step, for $T[i][j]$, we have $j - i - 1$ ways to choose an vertex $k$, so that breaks it into two or one subproblems. One subproblem is triangulate polygon defined by $i$th point to $k$th point. The other is triangulate polygon defined by $k$th point to $j$th point. Choose the smallest cost among them can guarantee the optimality of $T[i][j]$. Therefore the final answer $T[1][n]$ is optimal. 
	
	\item \textbf{Runtime Analysis}
	
	There're $n^2$ subproblems and each of them will cost $O(n)$ time to check each possible situation. Therefore, the overall runtime is $O(n^3)$.

\end{qparts}

\newpage
\section{Three Partition}
\begin{qparts}
	\item \textbf{Main Idea}
	
	Define subproblem as $X[i][s_1][s_2]$, which represents whether it's possible to divide first $i$ numbers into three groups, such that the sum of first group is $s_1$ and the sum of second group is $s_2$. Suppose the $i$th number is $A[i]$.
	The recursive relation is as follows:
	\[
		X[i][s_1][s_2] = X[i-1][s_1][s_2] \lor X[i-1][s_1 - A[i]][s_2] \lor X[i-1][s_1][s_2 - A[i]]
	\]
	
	 And the final solution is $X[n][\frac{total}{3}][\frac{total}{3}]$.

	\item \textbf{Proof of Correctness}
	
	In base case, $X[1][0][0]$, $X[1][A[1]][0]$ and $X[1][0][A[1]]$ are true. This is trivial. In induction step, for $X[i][s_1][s_2]$ we have three choices, i.g. put $i$th number into first group, second group or third group. $X[i-1][s_1][s_2]$ is the situation that put it into first group. $X[i-1][s_1 - A[i]][s_2]$ is the situation that put it into second group. $X[i-1][s_1][s_2 - A[i]]$ is the situation that put it into third group. So $X[i][s_1][s_2]$ will be optimal and the final answer will be optimal.
	
	\item \textbf{Runtime Analysis}
	
	There're $n(\sum a_i)^2$ subproblems and each of them will cost $O(1)$ time. Therefore, the overall runtime is $O(n(\sum a_i)^2)$.	

\end{qparts}

\newpage
\section{2-SAT}
\begin{qparts}
	
	\item 

	if $G_I$ has a strongly connected component containing both $x$ and $\neg x$ for some variable $x$. Then there's a path from $x$ to $\neg x$ and a path from $\neg x$ to $x$. The edges in this graph can be considered as implication, so we have $x \Rightarrow  \neg x$ and $\neg x \Rightarrow  x$ by transitive rule.
	So if $x$ is assigned with true, we have contradiction $true\Rightarrow false$. If  $x$ is assigned with false, we have $true\Rightarrow false$ as well. Therefore this problem has no valid solution.
	
	\item
	
	Note that the clause $(\alpha \lor \beta)$ is equivalent to $(\neg \alpha  \Rightarrow \beta) \land (\neg \beta\Rightarrow  \alpha)$. The edges added in to graph $G_I$ is symmetric. 
	
	Suppose SCC $A$ contains variables $v_1, v_2, v_3, \cdots v_n$. By symmetric property, there exists SCC $\neg A$ that contains variables $\neg v_1, \neg v_2, \neg v_3, \cdots \neg v_n$. Similarly, if there's an edge from SCC $A$ to SCC $B$, there exist an edge from SCC $\neg B$ to $\neg A$.
	
	In base case, there're only two SCCs and denote them as $A$ and $\neg A$. This is trivial. Assigned one of them with true and the other with false will satisfy requirement. In induction step, suppose SCC $A$ is a sink. Then SCC $\neg A$ must be a source. Suppose there's an edge from SCC $B$ to SCC $A$. Then there's an edge from SCC $\neg A$ to SCC $\neg B$. Assign $A$ with all trues can guarantee edge from SCC $B$ to SCC $A$ is satisfied. And assign $\neg A$ with all false can guarantee edge from SCC $\neg A$ to SCC $\neg B$ is satisfied. Therefore, delete all of them will not introduce conflicts and can always find a satisfaction solution.
	
	\item
	
	So first compute SCCs (meta-graph $G_M$ of $G_I$) and assign value according to the previous mentioned way. Compute meta graph will cost $(O(|E| + |V|))$ time and assign will cost linear time. So the total runtime is linear.
		
\end{qparts}






\end{document}