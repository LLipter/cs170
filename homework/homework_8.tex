% Search for all the places that say "PUT SOMETHING HERE".

\documentclass[11pt]{article}
\usepackage{amsmath,textcomp,amssymb,geometry,graphicx,enumerate}

\def\Name{Ran Liao}  % Your name
\def\SID{3034504227}  % Your student ID number
\def\Homework{8} % Number of Homework
\def\Session{Spring 2019}


\title{CS170--Spring 2019 --- Homework \Homework\ Solutions}
\author{\Name, SID \SID}
\markboth{CS170--\Session\  Homework \Homework\ \Name}{CS170--\Session\ Homework \Homework\ \Name}
\pagestyle{myheadings}
\date{\today}

\newenvironment{qparts}{\begin{enumerate}[{(}a{)}]}{\end{enumerate}}
\def\endproofmark{$\Box$}
\newenvironment{proof}{\par{\bf Proof}:}{\endproofmark\smallskip}

\textheight=9in
\textwidth=6.5in
\topmargin=-.75in
\oddsidemargin=0.25in
\evensidemargin=0.25in


\begin{document}
\maketitle
Collaborators:Jingyi Xu, Renee Pu

\section{Study Group}
	\begin{tabular}{ll}
		Name		&   SID         		\\\hline
		Ran Liao		&   3034504227  	\\  
		Jingyi Xu		&   3032003885  	\\
		Renee Pu		&   3032083302  	\\
	\end{tabular}

	



\newpage
\section{A Dice Game}
\begin{qparts}
	
	\item 
	
	\[
		W(x, y, z) = \max \{\frac{1}{6} W(x+1, y, 0)\}		
	\]
	
	\item 
	
	\renewcommand{\theenumii}{\roman{enumii}}
	\begin{enumerate}
		\item \textbf{Main Idea}
		
		Define subproblem $L[i]$ representing the minimum number of beams that can be welded into a $T$-foot beam. And we have the following recursive relation. And let $T[0] = 0$.
		
		\[
			L[i] = \min_{j=1}^{k}\{L[i - c_j] + 1\}
		\]
		
		\item \textbf{Proof of Correctness}
		
		The base case is obviously correct. Then, suppose we want to compute $L[i]$ and, $L[j]$ (where $j<i$) are optimal solutions. We use the minimum value of $L[j]$ for all possible $j$. This is actually greedy approach and thus guarantee the  optimality of $L[i]$. Therefore, $L[T]$ will be optimal solution.
		
		\item \textbf{Runtime Analysis}
		
		$T$ iterations is needed to solve $T$ subproblems and each of them will cost $O(k)$ time. Therefore, the overall runtime is $O(kT)$ and $O(T)$ space is needed.
	
	\end{enumerate}
		
\end{qparts}


\newpage
\section{Egg Drop}
\begin{qparts}
	\item 
	
	Drop the first egg from 14th story. If it doesn't break, drop it from $(14 + 13)$th story. If still it doesn't break, drop it from $(14 + 13 + 12)$th story. Repeat this process until the first egg breaks, and then use the second egg test remaining possible range one by one in increasing story order.
	
	14 drops is needed for this strategy. Actually, it doesn't depend on when the first egg breaks, 
	
	\item
	
	\renewcommand{\theenumii}{\roman{enumii}}
	\begin{enumerate}
		\item \textbf{Main Idea}
		
		Define subproblem $F[i][j]$ representing the number of drops needed for a $i$ story building and have $j$ eggs.
		The recursive relation is:		
		\[
			F[i][j] = \min_{x = 1}^{i}\{ \max\{ F[i - x][j] + 1 , F[x-1][j-1] + 1 \} \}
		\]
		
		\item \textbf{Proof of Correctness}
		
		There're two possible outcomes when dropping egg from $x$th story. If it doesn't break, the solution is between $(x+1)$th story and $i$th story. It can be considered as a new building with $i-x$ stories and can be represented like this $F[i - x][j] + 1$. If it breaks, the solution is between $1$st story and $(x-1)$th story. So it can be represented by $F[x-1][j-1] + 1$. The max operator will make sure we consider the worst case. The min operator will give the optimal strategy.
		
		\item \textbf{Runtime Analysis}
		
		There're $nk$ subproblems and each of them will cost $O(n)$ time. Therefore, the overall runtime is $O(n^2k)$.
	
	\end{enumerate}

\end{qparts}


\newpage
\section{Non-Prefix Code}
\begin{qparts}
	\item \textbf{Main Idea}
	
	Define subproblem as $N[i]$, which represents the number of ways we can interpret string $s[0 \cdots i]$. And the recursive relation is as follows. $N[0]$ is 0.
	
	\[
		N[i] = \sum_{(k, v) \in d} N[i - \operatorname{len}(v)] \operatorname{Match}(v, i)
	\]
	And
	\[ \operatorname{Match}(v, i) = 
	\left\{
		\begin{array}{rcl}
			1     &      & \text{string } s[i - \operatorname{len}(v) + 1 \cdots i] \text{ matches with } v  \\
			0      &      & \text{Otherwise}  \\
		\end{array}
	\right.
	\]
	\item \textbf{Proof of Correctness}
	
	The basic case is trivial and correct. Then, suppose we want to compute $N[i]$, and $N[j]$ (where $j<i$) are all optimal solutions. If character $v$ can be matched in $s[i - \operatorname{len}(v) + 1 \cdots i]$, there're $N[i - \operatorname{len}(v)]$ ways to interpret $s[1 \cdots i]$. Sum over all possible $v$ will make $N[i]$ optimal.
	
	
	\item \textbf{Runtime Analysis}
	
	This algorithm will solve $n$ subproblems and each of them will check $m$ possible values in $d$. Each check will cost at most $O(l)$ time. Therefore, the overall runtime is $O(nml)$.
	

\end{qparts}




\newpage
\section{Triangulating a polygon}
\begin{qparts}
	\item \textbf{Main Idea}
	
	Define subproblem as $X[i][s_1][s_2]$, which represents whether it's possible to divide first $i$ numbers into three groups, such that the sum of first group is $s_1$ and the sum of second group is $s_2$. Suppose the $i$th number is $A[i]$.
	The recursive relation is as follows:
	\[
		X[i][s_1][s_2] = X[i-1][s_1][s_2] \lor X[i-1][s_1 - A[i]][s_2] \lor X[i-1][s_1][s_2 - A[i]]
	\]
	
	 And the final solution is $X[n][\frac{total}{3}][\frac{total}{3}]$.

	\item \textbf{Proof of Correctness}
	
	In base case, $X[1][0][0]$, $X[1][A[1]][0]$ and $X[1][0][A[1]]$ are true. This is trivial. In induction step, for $X[i][s_1][s_2]$ we have three choices, i.g. put $i$th number into first group, second group or third group. $X[i-1][s_1][s_2]$ is the situation that put it into first group. $X[i-1][s_1 - A[i]][s_2]$ is the situation that put it into second group. $X[i-1][s_1][s_2 - A[i]]$ is the situation that put it into third group. So $X[i][s_1][s_2]$ will be optimal and the final answer will be optimal.
	
	\item \textbf{Runtime Analysis}
	
	There're $n(\sum a_i)^2$ subproblems and each of them will cost $O(1)$ time. Therefore, the overall runtime is $O(n(\sum a_i)^2)$.	

\end{qparts}

\newpage
\section{Three Partition}
\begin{qparts}
	\item \textbf{Main Idea}
	
	Define subproblem as $X[i][s_1][s_2]$, which represents whether it's possible to divide first $i$ numbers into three groups, such that the sum of first group is $s_1$ and the sum of second group is $s_2$. Suppose the $i$th number is $A[i]$.
	The recursive relation is as follows:
	\[
		X[i][s_1][s_2] = X[i-1][s_1][s_2] \lor X[i-1][s_1 - A[i]][s_2] \lor X[i-1][s_1][s_2 - A[i]]
	\]
	
	 And the final solution is $X[n][\frac{total}{3}][\frac{total}{3}]$.

	\item \textbf{Proof of Correctness}
	
	In base case, $X[1][0][0]$, $X[1][A[1]][0]$ and $X[1][0][A[1]]$ are true. This is trivial. In induction step, for $X[i][s_1][s_2]$ we have three choices, i.g. put $i$th number into first group, second group or third group. $X[i-1][s_1][s_2]$ is the situation that put it into first group. $X[i-1][s_1 - A[i]][s_2]$ is the situation that put it into second group. $X[i-1][s_1][s_2 - A[i]]$ is the situation that put it into third group. So $X[i][s_1][s_2]$ will be optimal and the final answer will be optimal.
	
	\item \textbf{Runtime Analysis}
	
	There're $n(\sum a_i)^2$ subproblems and each of them will cost $O(1)$ time. Therefore, the overall runtime is $O(n(\sum a_i)^2)$.	

\end{qparts}

\newpage
\section{2-SAT}
\begin{qparts}
	
	\item 

	if $G_I$ has a strongly connected component containing both $x$ and $\neg x$ for some variable $x$. Then there's a path from $x$ to $\neg x$ and a path from $\neg x$ to $x$. The edges in this graph can be considered as implication, so we have $x \Rightarrow  \neg x$ and $\neg x \Rightarrow  x$ by transitive rule.
	So if $x$ is assigned with true, we have contradiction $true\Rightarrow false$. If  $x$ is assigned with false, we have $true\Rightarrow false$ as well. Therefore this problem has no valid solution.
	
	\item
	
	Note that the clause $(\alpha \lor \beta)$ is equivalent to $(\neg \alpha  \Rightarrow \beta) \land (\neg \beta\Rightarrow  \alpha)$. The edges added in to graph $G_I$ is symmetric. 
	
	I want to prove there's no path between SCCs that contains a literal and its negation.
	Suppose variable $x$ is in SCC $A$ and variable $\neg x$ is in SCC $B$. And there's a path from $x$ to $\neg x$ that go through vertices $x, v_1, v_2, \cdots v_n, \neg x$.
	By symmetric property, there exists edges that from $x$ to $\neg v_{n}$, from $\neg v_n$ to $\neg v_{n-1}$, from $\neg v_n$ to $\neg v_{n-1} \cdots$, $\neg v_2$ to $\neg v_{1}$ and from $\neg v_1$ to $\neg x$. 
	Therefore, variable $v_i$ and $\neg v_i$ are in the same SCC which contradicts the assumption that none of $G_I$ ’s strongly connected components contain both a literal and its negation.
	
	Therefore, assign valid value in sink SCC will not affect the assignment of other SCCs. No contradiction will be introduced. And the method mentioned in this question is the easily way to make assignments.
	
	\item
	
	So first compute SCCs (meta-graph $G_M$ of $G_I$) and assign value according to the previous mentioned way. Compute meta graph will cost $(O(|E| + |V|))$ time and assign will cost linear time. So the total runtime is linear.
		
\end{qparts}






\end{document}