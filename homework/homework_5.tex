% Search for all the places that say "PUT SOMETHING HERE".

\documentclass[11pt]{article}
\usepackage{amsmath,textcomp,amssymb,geometry,graphicx,enumerate}

\def\Name{Ran Liao}  % Your name
\def\SID{3034504227}  % Your student ID number
\def\Homework{5} % Number of Homework
\def\Session{Spring 2019}


\title{CS170--Spring 2019 --- Homework \Homework\ Solutions}
\author{\Name, SID \SID}
\markboth{CS170--\Session\  Homework \Homework\ \Name}{CS170--\Session\ Homework \Homework\ \Name}
\pagestyle{myheadings}
\date{\today}

\newenvironment{qparts}{\begin{enumerate}[{(}a{)}]}{\end{enumerate}}
\def\endproofmark{$\Box$}
\newenvironment{proof}{\par{\bf Proof}:}{\endproofmark\smallskip}

\textheight=9in
\textwidth=6.5in
\topmargin=-.75in
\oddsidemargin=0.25in
\evensidemargin=0.25in


\begin{document}
\maketitle
Collaborators:Jingyi Xu, Renee Pu

\section{Study Group}
	\begin{tabular}{ll}
		Name		&   SID         		\\\hline
		Ran Liao		&   3034504227  	\\  
		Jingyi Xu		&   3032003885  	\\
		Renee Pu		&   3032083302  	\\
	\end{tabular}



\newpage
\section{Updating Labels}

\begin{qparts}
	\item \textbf{Main Idea}

	Run DFS from root node $r$ and update label whenever visit a new vertex. Suppose vertex $v$ is currently being visited and $l(v) = k$. The $k$th ancestor of $v$ is located on the $k$th element in stack counting from top (this value can be access in $O(1)$ time, if we use an array and a pointer to simulate a stack). Donate this vertex as $w$, then update $l(v) = l(w)$.
	
	\item \textbf{Proof of Correctness}
	
	Labels are updated from root $r$ to leaf nodes. Therefore the ancestor of a given vertex $v$ is updated before $v$. This property can guarantee the correctness of algorithm.
	

	\item \textbf{Runtime Analysis}
	
	DFS will cost $O(|V| + |E|)$.
	

\end{qparts}

\newpage
\section{Count Four Cycle}

\begin{qparts}
	\item \textbf{Main Idea}
	
	Suppose $A$ is the adjacency matrix of graph $G$. $A_{i,j} = 1$ iff there's an edge between vertex $i$ and $j$. Compute $A^2$ and then subtract 1 from $A^2_{i, i}$ for $ \forall A_{i, j} = 1$.  Then compute $A^3 = A^2 A$ and subtract 1 from $A^3_{i, j}$ for $\forall A_{j, k} = 1$. Lastly, compute $A^4 = A^3 A$. There's a four cycle iff $\exists A^4_{i, i} > 0$.

	\item \textbf{Proof of Correctness}
	
	$A$ represents the number of path between every pair of vertices with length 1. $A^2$ represents the number of path between every pair vertices with length 2, and so on. Therefore check $A^4{i, i}$ can reveal the existence of four cycles. All subtraction made previous is for eliminating invalid cycles.

	\item \textbf{Runtime Analysis}
	
	The trivial matrix product will cost $O(|V|^3)$ time, and all subtraction can be finished within $O(|V|^2)$ time. Therefore, the overall runtime is $O(|V|^3)$.
	

\end{qparts}

\newpage
\section{Constrained Dijkstra}

\begin{qparts}
	\item \textbf{Main Idea}
	
	Run Dijkstra algorithm on vertex $v_0$ and record shortest path in array $p$. Then reverse all edges in $G$, denote the new graph as $G_M$. Run Dijkstra algorithm on vertex $v_0$ again in new graph and record shortest path in array $p_M$. The shortest path between $u$ and $v$ can be reconstruct by combining path from $v_0$ to $u$ in $G_M$ and path from $v_0$ to $v$ in $G$.
	
	\item \textbf{Proof of Correctness}
	
	The path from $u$ to $v$ can be divided into two parts, namely, path from $u$ to $v_0$ and path from $v_0$ to $v$. The shortest path from $u$ to $v_0$ can be found in $G_M$. The shortest path from $v_0$ to $v$ can be found in $G$. 

	\item \textbf{Runtime Analysis}
	
	Dijkstra algorithm will cost $O((|V| + |E|)\log{|V|})$

\end{qparts}
\newpage
\section{Arbitrage}

\begin{qparts}
	\item
	
	\renewcommand{\theenumii}{\roman{enumii}}
	\begin{enumerate}
		\item \textbf{Main Idea}
		
		Construct a graph $G$ where $v_i$ represents currency $c_i$. The weight of edge between $v_i$ and $v_j$ will be $\frac{1}{r_{i,j}}$. Run a modified Bellman-Ford on this graph start with vertex $s$. The update role is changed to $\operatorname{dist}(v) = \operatorname{min}(\operatorname{dist}(v), \operatorname{dist}(u) \times l(u, v))$. Initially, $\operatorname{dist}(s) = 1$.
		
		\item \textbf{Proof of Correctness}
		
		The multiplication and addition has similar associative and commutative property. All deduction for Bellman-Ford algorithm will still holds true for the modified version. In modified version, edge with weight less than 1 will be consider as a ``negative" edge.
		
		\item \textbf{Runtime Analysis}
		
		The modification will not change Bellman-Ford algorithm's runtime. Therefore it is $O(|V||E|)$.
		
	\end{enumerate}
	
	\item
	
		\renewcommand{\theenumii}{\roman{enumii}}
	\begin{enumerate}
		\item \textbf{Main Idea}
		
		Use the same graph defined in part (a) and add additional iteration to the outer loop. If some vertices is updated in the final iteration, arbitrage situation exists.
		
		\item \textbf{Proof of Correctness}
		
		If there's an arbitrage situation, there must exist a loop where weights' product is less than 1. This is similar to have a negative loop in the original version of Bellman-Ford algorithm. This ``negative" loop will cause the shortest path never stop updating.
		
		\item \textbf{Runtime Analysis}
		
		The modification will not change Bellman-Ford algorithm's runtime. Therefore it is $O(|V||E|)$.
		
	\end{enumerate}

\end{qparts}
\newpage
\section{Bounded Bellman-Ford}

\begin{qparts}
	\item \textbf{Pseudocode}

	Modified-Bellman-Ford($G = (V, E)$)\{

		\qquad for $\forall u \in V$:
		
		\qquad\qquad $\operatorname{dist}(u) = \infty$
		
		\qquad\qquad $\operatorname{prev}(u) = \operatorname{nil}$
		
		\qquad $\operatorname{dist}(s) = 0$
	
		\qquad for $i$ from 1 to $k$:
		
		
		\qquad\qquad for $\forall e=(u,v) \in E$:
		
		\qquad\qquad\qquad if($\operatorname{dist}(u) + l(u, v) < \operatorname{dist}(v$))
		
		\qquad\qquad\qquad\qquad  $\operatorname{dist}(v) = \operatorname{min}(\operatorname{dist}(v), \operatorname{dist}(u) + l(u, v))$
		
		\qquad\qquad\qquad\qquad $\operatorname{prev}(v) = u$
	
	\}
		
	\item \textbf{Proof of Correctness}
	
	Each iteration in outer loop will try to consider the shortest path with 1 more edges. Therefore, after $k$ iterations, the dist will contain information about the shortest path with no more than k edges.

	\item \textbf{Runtime Analysis}
	
	The outer loop will run $k$ times, therefore the runtime is $O(k|E|)$.
	

\end{qparts}
     


\end{document}