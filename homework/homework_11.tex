% Search for all the places that say "PUT SOMETHING HERE".

\documentclass[11pt]{article}
\usepackage{amsmath,textcomp,amssymb,geometry,graphicx,enumerate}

\def\Name{Ran Liao}  % Your name
\def\SID{3034504227}  % Your student ID number
\def\Homework{11} % Number of Homework
\def\Session{Spring 2019}


\title{CS170--Spring 2019 --- Homework \Homework\ Solutions}
\author{\Name, SID \SID}
\markboth{CS170--\Session\  Homework \Homework\ \Name}{CS170--\Session\ Homework \Homework\ \Name}
\pagestyle{myheadings}
\date{\today}

\newenvironment{qparts}{\begin{enumerate}[{(}a{)}]}{\end{enumerate}}
\def\endproofmark{$\Box$}
\newenvironment{proof}{\par{\bf Proof}:}{\endproofmark\smallskip}

\textheight=9in
\textwidth=6.5in
\topmargin=-.75in
\oddsidemargin=0.25in
\evensidemargin=0.25in


\begin{document}
\maketitle
Collaborators:Jingyi Xu, Renee Pu

\section{Study Group}
	\begin{tabular}{ll}
		Name		&   SID         		\\\hline
		Ran Liao		&   3034504227  	\\  
		Jingyi Xu		&   3032003885  	\\
		Renee Pu		&   3032083302  	\\
	\end{tabular}

	
\newpage
\section{Bipartite Vertex Cover}
\begin{qparts}
	\item \textbf{Main Idea}
	
	Since $G$ is a bipartite graph, we can split $V$ into two disjoint set $L$ and $R$ such that there's no edge linking two vertices in the same set.
	
	Then, run max-flow algorithm on following graph $G^\prime$. Add nodes $s$ and $t$ into the graph. Add a directed edge from $s$ to every node in $L$ and from every node in $R$ to $t$. Convert the undirected edges between $L$ and $R$ to directed edges. Set the capacity of every edge to 1. As textbook mentioned, this max-flow can be converted into a max-matching.
	
	A minimum vertex cover can be constructed as follows. Let $U$ be the set of unmatched vertices in $L$, and $Z$ be the set of vertices that are either in $U$ or are connected to $U$ by alternating paths(a path of odd length that starts and ends with a non-covered vertex, and whose edges alternate between matched edges and unmatched edges). Let $K = (L \setminus Z) \cup (R \cap Z)$. Z is the minimum vertex cover we're looking for.
	
	\item \textbf{Proof of Correctness}
	
	Suppose that $M$ is a maximum matching. No vertex in a vertex cover can cover more than one edge of $M$. So $|M|$ is a lower bound for vertex cover and if we can construct a vertex cover with $|M|$ vertices, it must be a minimum cover.
	
	Every edge $e$ in $E$ either belongs to an alternating path, or it has a left endpoint in $K$. If $e$ is matched but not in an alternating path, then its left endpoint cannot be in an alternating path (because two matched edges can not share a vertex) and thus belongs to $(L \setminus Z)$. Alternatively, if $e$ is unmatched but not in an alternating path, then its left endpoint cannot be in an alternating path, for such a path could be extended by adding $e$ to it. Thus, $K$ forms a vertex cover with size $|M|$.
	
\end{qparts}

\newpage
\section{Direct Bipartite Matching}
\begin{qparts}
	\item \textbf{Main Idea}
	
	To prove $M$ is a maximum matching if and only if there does not exist an alternating path, it's equivalent to prove $M$ is \textbf{not} a maximum matching if and only if there does \textbf{exist} an alternating path.
	
	Suppose $M$ is not a maximum matching, run the max-flow algorithm on this graph. At least one augment path can be found. Denote it as $s \to v_1 \to v_2 \to \cdots \to v_{2n} \to t$. When $i$ is odd, $v_i \in L$ and $v_i \to v_{i+1} \in E \setminus M$, otherwise $v_i \in R$ and $v_i \to v_{i+1} \in M$. The number of edges in this path is odd. Therefore, it is an alternating path.
	
	Suppose there does exist an alternating path. Denote it as $v_1 \to v_2 \to \cdots \to v_{2n}$. By definition, $(v_1, v_2), (v_3, v_4) \cdots (v_{2n-1}, v_{2n}) \in E \setminus M$ and denote it as $E_1$. Denote the remaining edges as $E_2$. We can construct a larger flow $M^\prime$ as follows.
	
	\[
		M^\prime = (M \cup E_1) \setminus E_2
	\]
	
	Therefore, $M$ cannot be a maximum matching.
	
	\item
	\renewcommand{\theenumii}{\roman{enumii}}
	\begin{enumerate}
		\item \textbf{Main Idea}
		
		Create a dumb node $s$ and add edge $(s, v)$ if $v$ is an unmatched vertex. Use a global variable $depth$ to record the depth of the search tree. If $depth$ is odd, only unmatched edges can be explored. Otherwise only matched edges can be explored. Run BFS starting from $s$.
		
		\item \textbf{Proof of Correctness}
		
		The alternating path will start from an unmatched point and alternating between matched and unmatched edges. The variable $depth$ will be helpful to determine whether to explore matched edges or unmatched edges.
		
		\item \textbf{Runtime Analysis}
		
		The modified BFS will not change overall runtime. Therefore it is $O(|V| + |E|)$.
		
	\end{enumerate}
	
	\item
	\renewcommand{\theenumii}{\roman{enumii}}
	\begin{enumerate}
		\item \textbf{Main Idea}
		
		Find an alternating path use the algorithm in part b and construct a larger matching use method mentioned in part a.
		Keep doing this until there's no alternating path exists.
		
		\item \textbf{Proof of Correctness}
		
		As proved in part a, $M$ is a maximum matching if and only if there does not exist an alternating path. And if there's an alternating path, part b algorithm will find it.
		
		\item \textbf{Runtime Analysis}
		
		The algorithm in part b will be run at most $O(|V|)$ times, and each run will cost $O(|V| + |E|)$.
		Therefore, the overall runtime is  $O((|V| + |E|)|V|) = O(|V||E|)$.
		
	\end{enumerate}
	
\end{qparts}


\newpage
\section{Zero-Sum Battle}
\begin{qparts}
	
	\item 
	
	\[
		\max p
	\]
	\[
		p \le -10x_1 +4x_2 + 6x_3 \text{ (payoff when trainer B chooses the ice Pokemon)}
	\]
	\[
		p \le 3x_1 -1x_2 - 9x_3 \text{ (payoff when trainer B chooses the water Pokemon)}
	\]
	\[
		p \le 3x_1 -3x_2 + 2x_3 \text{ (payoff when trainer B chooses the fire Pokemon)}
	\]
	\[
		x_1 + x_2 + x_3 = 1
	\]
	\[
		x_1 \ge 0
	\]
	\[
		x_2 \ge 0
	\]
	\[
		x_3 \ge 0
	\]
	
	The optimal strategy is $(0.335, 0.563, 0.102)$ and the payoff is $-0.48$.
		
	\item 
	
	\[
		\min p
	\]
	\[
		p \ge -10y_1 + 3y_2 + 3y_3 \text{ (payoff when trainer A chooses the dragon Pokemon)}
	\]
	\[
		p \ge 4y_1 -1y_2 - 3y_3 \text{ (payoff when trainer A chooses the steel Pokemon)}
	\]
	\[
		p \ge 6y_1 -9y_2 + 2y_3 \text{ (payoff when trainer A chooses the rock Pokemon)}
	\]
	\[
		y_1 + y_2 + y_3 = 1
	\]
	\[
		y_1 \ge 0
	\]
	\[
		y_2 \ge 0
	\]
	\[
		y_3 \ge 0
	\]
	
	The optimal strategy is $(0.268, 0.323, 0.409)$ and the payoff is $-0.48$.

\end{qparts}



\newpage
\section{Domination}
\begin{qparts}
	\item 
	
	It should be 0 since choosing E instead will always give a better payoff.
	
	\item
	
	It should also be 0 since choosing B instead will always give a better payoff(column player wants to minimize the payoff).
	
	\item
	
	Both of them should be $(0.5, 0.5)$, since they are completely symmetric
	
\end{qparts}







\end{document}