% Search for all the places that say "PUT SOMETHING HERE".

\documentclass[11pt]{article}
\usepackage{amsmath,textcomp,amssymb,geometry,graphicx,enumerate}

\def\Name{Ran Liao}  % Your name
\def\SID{3034504227}  % Your student ID number
\def\Homework{11} % Number of Homework
\def\Session{Spring 2019}


\title{CS170--Spring 2019 --- Homework \Homework\ Solutions}
\author{\Name, SID \SID}
\markboth{CS170--\Session\  Homework \Homework\ \Name}{CS170--\Session\ Homework \Homework\ \Name}
\pagestyle{myheadings}
\date{\today}

\newenvironment{qparts}{\begin{enumerate}[{(}a{)}]}{\end{enumerate}}
\def\endproofmark{$\Box$}
\newenvironment{proof}{\par{\bf Proof}:}{\endproofmark\smallskip}

\textheight=9in
\textwidth=6.5in
\topmargin=-.75in
\oddsidemargin=0.25in
\evensidemargin=0.25in


\begin{document}
\maketitle
Collaborators:Jingyi Xu, Renee Pu

\section{Study Group}
	\begin{tabular}{ll}
		Name		&   SID         		\\\hline
		Ran Liao		&   3034504227  	\\  
		Jingyi Xu		&   3032003885  	\\
		Renee Pu		&   3032083302  	\\
	\end{tabular}

	
\newpage
\section{Bipartite Vertex Cover}
\begin{qparts}
	\item \textbf{Main Idea}
	
	Build the residual network based on maximum flow $f$. Then find a path from $t$ to $s$ that go through edge $(v, u)$ in residual network. Push 1 unit flow into this path and fix the wrong capacity from $c_{uv}$ to $c_{uv} - 1$. Then run the original max-flow algorithm starting from this residual network.

	\item \textbf{Proof of Correctness}
	
	Pushing 1 unit back from $t$ to $s$ that go through edge $(v, u)$ will make the flow go through edge $(u, v)$ be valid. Then run the original max-flow algorithm again from this is sure will give an optimal solution.
	
	\item \textbf{Runtime Analysis}
	
	Build residual network will cost $O(|E|)$ time. Find a path from $t$ to $s$ that go through edge $(v, u)$ by DFS or BFS will cost $O(|V| + |E|)$ time. Then push 1 unit into this path will cost at most $|E|$ time. Since the max-flow solution after repairing capacity of edge $(u, v)$ cannot be larger than the original one. At most 1 iteration is needed if run max-flow algorithm starting from this. So total runtime is still linear, which is $O(|V| + |E|)$.

\end{qparts}


\newpage
\section{Zero-Sum Battle}
\begin{qparts}
	
	\item 
	
	\[
		\max p
	\]
	\[
		p \le -10x_1 +4x_2 + 6x_3 \text{ (payoff when trainer B chooses the ice Pokemon)}
	\]
	\[
		p \le 3x_1 -1x_2 - 9x_3 \text{ (payoff when trainer B chooses the water Pokemon)}
	\]
	\[
		p \le 3x_1 -3x_2 + 2x_3 \text{ (payoff when trainer B chooses the fire Pokemon)}
	\]
	\[
		x_1 + x_2 + x_3 = 1
	\]
	\[
		x_1 \ge 0
	\]
	\[
		x_2 \ge 0
	\]
	\[
		x_3 \ge 0
	\]
	
	The optimal strategy is $(0.335, 0.563, 0.102)$ and the payoff is $-0.48$.
		
	\item 
	
	\[
		\min p
	\]
	\[
		p \ge -10y_1 + 3y_2 + 3y_3 \text{ (payoff when trainer A chooses the dragon Pokemon)}
	\]
	\[
		p \ge 4y_1 -1y_2 - 3y_3 \text{ (payoff when trainer A chooses the steel Pokemon)}
	\]
	\[
		p \ge 6y_1 -9y_2 + 2y_3 \text{ (payoff when trainer A chooses the rock Pokemon)}
	\]
	\[
		y_1 + y_2 + y_3 = 1
	\]
	\[
		y_1 \ge 0
	\]
	\[
		y_2 \ge 0
	\]
	\[
		y_3 \ge 0
	\]
	
	The optimal strategy is $(0.268, 0.323, 0.409)$ and the payoff is $-0.48$.

\end{qparts}



\newpage
\section{Domination}
\begin{qparts}
	\item 
	
	It should be 0 since choosing E instead will always give a better payoff.
	
	\item
	
	It should also be 0 since choosing B instead will always give a better payoff(column player wants to minimize the payoff).
	
	\item
	
	Both of them should be $(0.5, 0.5)$, since they are completely symmetric
	
\end{qparts}







\end{document}